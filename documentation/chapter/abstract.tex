%!TEX root = ../documentation.tex
\chapter{Abstract}
\label{ch:abstract}

Covid-19 ist eine schwerwiegende Erkankung, welche bereits auf der ganzen Welt eine wichtige Rolle spielt und nicht nur das Leben vieler Leute einschränkt, sondern auch das von vielen gefährdet. Es sind bereits knapp 35 Millionen Leute an Covid-19 erkrankt worden und von diesen sind etwa 1 Millionen gestorben. Um diese Zahlen zu verringern, ist es wichtig die Erkrankung schnell zu identifizieren und diagnostizieren. Es gibt hierfür bereits einige Tests, jedoch haben Studien herausgefunden, dass an Covid-19 erkrankte Patienten, ein abnormales Röntgenbild der Lunge wieder geben. Da Röntgenbilder eine der schnellsten Möglichkeiten sind, um Covid-19 fest zu stellen, ist es wichtig, dies weiter zu verfeinern. Inspiriert durch diese Idee und auch von vorherigen Arbeiten, möchten wir hier ein Deep Learning Modell konstruieren, welches automatisch, so gut wie möglich, erkennt ob ein Patient Covid-19 hat, oder nicht. Hierfür haben wir ein Datenset übergeben bekommen, welches wir dann bearbeiten, um die Bilder von Unreinheiten zu befreien und im folgenden die größe des Datensets zu vergrößern, so dass mehr Bilder zum Trainieren und Testen benutzt werden können. Nachdem das ganze mit verschiedenen Modellen getestet und verglichen wurde, sind wir auf eine Genauigkeit von etwa 92\% (+/- 5\%) gekommen, was schon ein zufrieden stellendes Ergebnis ist, man jedoch immer noch weiter ausbessern kann. Im folgenden wird dann das ganze Vorgehen, inklusive der Modelle erklärt.