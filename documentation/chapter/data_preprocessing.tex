%!TEX root = ../documentation.tex
\chapter{Datenvorverarbeitung}
\label{ch:data_preprocessing}

\section{Bilddaten}

In der vorherigen Datenanalyse wurden folgende Probleme herausgestellt:

\begin{enumerate}
	\item{Dimensionsunterschiede}
	\item{Farbunterschiede}
	\item{Kontrastunterschiede}
\end{enumerate}

Diese sollen nun mit geeigneten Vorverarbeitungsschritten gelöst werden.

Die Bilder müssen als Eingabe für das Model in einer einheitlichen Größe vorliegen. Dazu skalieren wir alle Bilder auf den Schwerpunkt der Dimensionen, welcher bei 320 x 320 Pixeln liegt.
Nur drei der Bilder sind kleiner als 320 x 320 Pixel und mussen damit vergrößert werden um die gewünschte Dimension zu erreichen. Bei einer Vergrößerung können Artifakte im Bild auftreten, welche das Model beeinflussen können. Aufgrund der geringen Anzahl können diese in unserem Fall vernachlässigt werden.\\
Farbliche Informationen spielen in den Röntgenbildern keine Rolle. Daher verwerfen wir die Farbinformationen und wandeln alle Farbbilder in Schwarzweißbilder um.\\
Weiterhin werden die Kontrastunterschiede über einen Histogrammausgleich verringert. Dadurch werden die Bilder, welche aus vielen verschiedenen Quellen stammen, homogenisiert.

\begin{figure}[H]
	\centering
	\begin{subfigure}[b]{0.45\textwidth}
		\centering
		\includegraphics[width=0.5\textwidth]{../images/42081.jpg}
		\caption{Originalbild 42081.jpg\\1524 x 1516 Pixel}
	\end{subfigure} \hfill
	\begin{subfigure}[b]{0.45\textwidth}
		\centering
		\includegraphics[width=0.5\textwidth]{../train_data/42081.jpg}
		\caption{Vorverarbeitetes Bild\\320 x 320 Pixel}
	\end{subfigure} \hfill
	\caption{Vergleich eines Originalbildes mit dem daraus resultierenden vorverarbeiten Bild}
\end{figure}