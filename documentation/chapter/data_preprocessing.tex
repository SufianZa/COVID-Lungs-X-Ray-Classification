%!TEX root = ../documentation.tex
\chapter{Datenvorverarbeitung}
\label{ch:data_preprocessing}

\section{Bilddaten}

In der vorherigen Datenanalyse wurden folgende Probleme herausgestellt:

\begin{enumerate}
	\item{Dimensionsunterschiede}
	\item{Farbunterschiede}
	\item{Kontrastunterschiede}
\end{enumerate}

Diese sollen nun mit geeigneten Vorverarbeitungsschritten gelöst werden.

Die Bilder müssen als Eingabe für das Model in einer einheitlichen Größe vorliegen. Dazu skalieren wir alle Bilder auf den Mittelwert der Dimension von 384x384 Pixeln.\\
Da die farblichen Informationen in Röntgenbildern irrelevant sind, werden alle Bilder zu Schwarzweißbildern umgewandelt.\\
Weiterhin werden die Kontrastunterschiede über einen Histogrammausgleich verringert.

\begin{figure}[H]
	\centering
	\begin{subfigure}[b]{0.45\textwidth}
		\includegraphics[width=\textwidth]{../images/42081.jpg}
		\caption{Originalbild 42081.jpg\\1524 x 1516 Pixel}
	\end{subfigure} \hfill
	\begin{subfigure}[b]{0.45\textwidth}
		\includegraphics[width=\textwidth]{../train_data/42081.jpg}
		\caption{Vorverarbeitetes Bild\\384 x 384 Pixel}
	\end{subfigure}
	\caption{Vergleich eines Originalbildes mit dem daraus resultierenden vorverarbeiten Bild}
\end{figure}